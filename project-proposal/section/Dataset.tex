\documentclass[../proposal.tex]{subfiles}

\begin{document}
\indent In our project, we aim to address the open problem of optical character recognition. More specfically, we want to address the open problem of converting handwritten mathematical expressions into LaTeX. To do so, we use ICDAR's 2023 Competition on Recognition of Handwritten Mathematical Expressions (CHROME) dataset. This dataset is partitioned into three tasks: online recognition, offline recognition, and bimodal recognition. Online recognition attempts to determine the characters as they are being written. For this purpose, the partitioned dataset for this task includes inkml files, a file format that stores the vectorized data of each expression written by hand, from previous iteratoins of CHROME\@. In our case, we will likely use the data set aside for task two, offline recognition. This partition of the dataset incudes rendered inkml files and scanned immages. It is meant for models that give the LaTeX expression for a written expression based on the entire expression rather than individual strokes. From the above explanation, it should be obvious bimodal deals with both. Also, it should be clear we are choosing offline recognition given the apparent complexity of online recognition. 

\indent While the general problem seems simple, it has several components. A given model has to identify the strokes of a given character, the positioning of the character, and the relationship between the characters to be able to find a suitable LaTeX expression. Based on the work of Sakshi and Kukreja, there are five main areas of challenges with current approaches: Preprocessing, Input/Output formats and representations, Recongnition model challenges, Compartive analysis and performance evaluation, and Hardware challenges. Based on the content of the class, we can ignore challenges pertaining to hardware, and Input/ Output formats and representation. Between the final three areas, we are narrowing down the exact question we would like to answer. Currently, we believe answering questions relating to the Comparative analysis such as developing guidelines for metric standardization and "forgiven"symbols based on similiarity. Alternatively, we could deal with recognition model challenges by comparing different neural network structures or determining and comparing optimal hyperparameters between models. Finally, we could consider the effect of multiple preprocessing techniques such as illumination, curve and blur, low resolution, noice, and occlusion on model accuracy. For our final project, we choose the final category, considering the effect of preprocessing techniques on model accuracy. 
\end{document}
