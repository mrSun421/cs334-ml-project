\documentclass[../report.tex]{subfiles}

\begin{document}
Predicting the weather has always been an important task in everyday life.
From a simple decision of bringing an umbrella, to more complicated decisions of watering crops for agriculture or evacuating citizens for a flood, predicting precipitation is an important objective.
Prior studies into predicting rainfall have been successful.
The studies used models and techniques such as SVMs, Extreme Gradient Boosting, Decision Trees, and LSTMs. \cite*{liyew_machine_2021,barrera-animas_rainfall_2022,rahman_rainfall_2022}
We planned to use similar models in our study.

Our study is using the NOAA quality controlled datasets \cite*{diamond_us_2013} as well U.S. Local Climatological Data from NOAA\cite*{NOAA_local_1996}.
The quality controlled dataset is split up into monthly, daily, hourly, and sub-hourly datasets.
Each dataset collects various atmospheric and earth data from various weather stations across the US.
Similarly, the USLCD does the same, with more detail, but much more missing data.
Prior studies into predicting precipitation used only one source of data.
Our study is unique in attempting to augment the quality controlled datasets with more features.
We specifically focus on data collected from Brunswick, GA.

\end{document}
