\documentclass[../report.tex]{subfiles}

\begin{document}
Predicting the weather has always been an important task in everyday life.
From a simple decision of bringing an umbrella, to more complicated decisions of watering crops for agriculture or evacuating citizens for a flood, predicting precipitation is an important objective.
Prior studies into predicting rainfall have been successful.
The studies used models and techniques such as SVMs, Extreme Gradient Boosting, Decision Trees, and LSTMs. \cite*{liyew_machine_2021,barrera-animas_rainfall_2022,rahman_rainfall_2022}
We planned to use similar models in our study.
The studies, however, usually only focus on data from a single location.

Our study is using the NOAA quality controlled datasets. \cite*{diamond_us_2013}
The dataset is split up into monthly, daily, hourly, and sub-hourly datasets.
Each dataset collects various atmospheric and earth data from various weather stations across the US.
Compared to the previous studies, we attempted to predict precipitation values from a variety of locations and air and ground data.
This is a novel approach, since we are generalizing precipitation to not a single location, but based on many locations.

\end{document}
