\documentclass[../report.tex]{subfiles}

\begin{document}
Precipitation prediction is a high priority for many groups for people from government entities to private corporations due to its far reaching consequences. Understanding when and how much precipitation will occur at a gvein location can aid conservation efforts, improving infrastructure, and allow for more efficient resource allocation for emergencies. Prior studies developing rainfall prediction models have fallen into two main categories: statistical and numerical. For our purposes, we consider only statistical models, which utilize machine learning to predict precipitation type or amount based on past weather data. 
The studies used models and techniques such as SVMs, Extreme Gradient Boosting, Decision Trees, and LSTMs. \cite*{liyew_machine_2021,barrera-animas_rainfall_2022,rahman_rainfall_2022}
We planned to use similar models in our study.

Our study is using the NOAA quality controlled datasets \cite*{diamond_us_2013} as well U.S. Local Climatological Data from NOAA\cite*{NOAA_local_1996}.
The quality controlled dataset is split up into monthly, daily, hourly, and sub-hourly datasets.
Each dataset collects various atmospheric and earth data from various weather stations across the US.
Similarly, the USLCD does the same, with more detail, but much more missing data.
Prior studies into predicting precipitation used only one source of data.
Our study is unique in attempting to augment the quality controlled datasets with more features.
We specifically focus on data collected from Brunswick, GA.

\end{document}
