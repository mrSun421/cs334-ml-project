\documentclass[../report.tex]{subfiles}

\begin{document}
We first needed to preprocess our data.
We first took the quality controlled data, and filtered the data to get rid of bad data.
Conveniently, packaged with the data, were instructions on certain fields.
For example, \verb|ST_FLAG| is a field described such that, when the value is greater than 0, an error has occurred in the raw data gathering.
Similarly, certain numeric fields, such as \verb|P_PRECIP|, an extremely low value is written to the field if it is missing.

Continuing, certain fields are non-numeric, and needed to be removed, such as those fields.
The fields removed without computation were

\begin{itemize}
    \item \verb|WBANNO|
    \item \verb|LST_DATE|
    \item \verb|LST_TIME|
    \item \verb|CRX_VN|
    \item \verb|SUR_TEMP_TYPE|
    \item \verb|SOLARAD_FLAG|
    \item \verb|SOLARAD_FLAG|
    \item \verb|SOLARAD_MAX_FLAG|
    \item \verb|SOLARAD_MIN_FLAG|
    \item \verb|SUR_TEMP_FLAG|
    \item \verb|SUR_TEMP_MAX_FLAG|
    \item \verb|SUR_TEMP_MIN_FLAG|
\end{itemize}
Clearly the flags were removed, as they only gave error information.
The station name is the same for all pieces of data.
We removed the local time because we already have UTC time to extract information.
To extract more features, we converted \verb|UTC_DATE| and \verb|UTC_TIME| to a python \verb|datetime| column labeled \verb|DATE|.

For the LCD dataset, a similar process was followed.
Labels that were strictly non-numeric were removed.
Note that LCD dataset took measurements every 20 minutes, so we had to only take every third bit of data.
\verb|DATE| was created by then rounding the date to the nearest hour.
We then merged the two datasets by matching the \verb|DATE| columns of the datasets.

After doing so, we then created new features based off of the year, month, day, and hour.
If there was missing data, for example, a missing precipitation value, we would randomly choose a non-missing value of the same day, month, and hour, but on a different year.
This is a reasonable assumption that the weather will do similar things on the same date on different years.

We then did a Pearson Correlation Matrix to see which features were correlated.
\begin{figure}
    \centering

    \label{fig:pearson_corr}
\end{figure}
From the Pearson Correlation Matrix we removed features that were correlated greater than 0.5.


\end{document}
